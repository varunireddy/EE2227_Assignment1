\documentclass{beamer}
\usepackage{ctex, hyperref}
\usepackage{calligra}
\usepackage[T1]{fontenc}
\usepackage{mathtools}
\usepackage{amsmath}
\usepackage{graphicx}
\author{EE18BTECH11005}
\title{CONTROL SYSTEMS - EE2227}
\subtitle{GATE - 2019 problem}
\institute{B.VARUNI}
\date{}
\usepackage{Tsinghua}

\begin{document}


\begin{frame}
	\titlepage
\end{frame}
\begin{frame}
     \frametitle{\textbf{QUESTION - 42, EC:}}
     {Consider a unity feedback system as shown in the figure,shown with an integral compensator k/s and open-loop transfer function} 
     \newline
    \newline G(s) = $\frac{1}{s^2+3s+2}$
   
    \newline
   
    where k>0. The positive value of k for which there are two poles of unity feedback system on j${\omega}$ {axis is equal to-----(rounded off to two decimal places)}
\end{frame}
\begin{frame}
 \begin{figure}
     \includegraphics[width=\linewidth]{gate.png}
    \end{figure}
\end{frame}
\begin{frame}{Transfer function of Negative feedback}
A transfer function is the relative function between input and output.
\newline In a negative feedback system an intermediate signal is defined as Z.
\begin{figure}
    \centering
    \includegraphics[width =\linewidth]{feedback.png}
    \caption{Caption}
    \label{fig:my_label}
\end{figure}
\end{frame}
\begin{frame}
\newline 
Y(s) = Z(s).G(s)
\newline 
Z(s) = X(s) - Y(s).H(s) => X(s) = Z(s)+Y(s).H(s)
\newline
X(s) = Z(s)+Z(s).G(s).H(s)
\newline 
$\frac{Y(s)}{X(s)}$ = $\frac{Z(s).G(s)}{Z(s)+Z(s).G(s).H(s)}$

So,the transfer function of negative feedback is $\frac{G(s)}{1+G(s).H(s)}$
\newline Since unit feedback H(s) = 1
\newline Now the transfer function of unity negative feedback is $\frac{G(s)}{1+G(s)}$
\end{frame}
\begin{frame}{Net transfer function}
  The net transfer function in the given question is.....
  \newline
  
  $\frac{Y(s)}{X(s)}$ = $\frac{G(s)*k/s}{1+G(s)*k/s}$
  \newline
  
  The characteristic equation is 1 + (G(s)x$\frac{k}{s}$) = 0
 \newline that is..,
 \newline 
 
 C.E = 1 + $\frac{k}{s(s^2+3s+2)}$ = 0
 =>s(s^2+3s+2) + k = 0
 \newline 
 
=> s^3+3s^2+2s+k=0
 \end{frame}
 \begin{frame}{Routh-Hurwitz Criterion}
 This criterion is based on arranging the coefficients of characteristic equation intoan array called Routh array.
 \newline
 
 q(s) = $a_0$s^n+$a_1$s^{n-1}+$a_2$s^{n-2}+.....+$a_{n-1}$s+$a_n$ = 0
\newline 

\begin{vmatrix}s^n\\s^{n-1}\\s^{n-2} \\ \vdots \end{vmatrix} \begin{vmatrix}
a_0 & a_2 & a_4 & \cdots \\
a_1 & a_3 & a_5 & \cdots  \\
b_1 & b_2 & b_3 & \cdots \\
\vdots & \vdots & \vdots & \ddots &\vdots 
 \cdots \\ \end{vmatrix} where b_1 =\frac{ a_1a_2-a_0a_3}{a_1}  \hspace{5pt} b_2 =\frac{ a_1a_4-a_0a_5}{a_1} \hspace{5pt} c_1=\frac{ b_1a_3-a_1b_2}{b_1}  \hspace{5pt}     c_2=\frac{ b_1a_5-a_1b_3}{b_1}  
 \end{frame}
 \begin{frame}
 For poles to lie on imaginary axis any one entire row of hurwitz matrix should be zero.
 \newline 
 
 For the given characteristic equation =  s^3+3s^2+2s+k = 0
 \newline \begin{vmatrix}s^3\\s^2\\s^1 \\ s^0 \end{vmatrix} \begin{vmatrix}
1 & 2 \\ 3 & k \\  \frac{6-k}{3} & 0\\ k & 0
\end{vmatrix}
\newline $For poles on $j$\omega$ axis any one of the row should be zero
\newline => $\frac{6-k}{3}$ = 0 or k = 0
\newline But given k>0 ...
\newline therefore, 6-k=0 => k = 6
\end{frame} 
\begin{frame}
To find the location of poles on j$\omega$ axis 
\newline 

Auxillary equation of the given CE is 3s^2 + k = 0
\newline
=> 3s^2 + 6 = 0
\newline
=> s = \pm j2
    
\end{frame}     
\end{document}
